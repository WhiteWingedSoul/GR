% \pagebreak[4]
% \hspace*{1cm}
% \pagebreak[4]
% \hspace*{1cm}
% \pagebreak[4]

\chapter{Đặt vấn đề và định hướng giải pháp} \label{problem-state}
\section{Đặt vấn đề}
Sự phát triển mạnh mẽ của khoa học công nghệ đi kèm với sự phát triển của Internet và điện thoại thông minh. Với lợi thế nhỏ gọn, tiện dụng và thông minh, smartphone đã trở thành một phần không thể tách biệt trong cuộc sống, hỗ trợ rất nhiều cho con người trong các hoạt động hằng ngày. Người dùng smartphone ngoài chức năng liên lạc, họ còn sử dụng nhiều loại dịch vụ khác nhau như giải trí, định vị, mua sắm, thanh toán trực tuyến,... Một trong những công dụng hữu ích phải kể đến đó là việc học tập trên điện thoại.

Nhiều chuyên gia nhận định, cùng với sự phát triển của Internet, Giáo dục trực tuyến (E-Learning) đang dần trở nên phổ biến nhờ tính tiện dụng, tương tác cao và nhu cầu rất lớn từ cộng đồng học tập. Với chiếc smartphone trên tay, chỉ bằng một vào thao tác tìm kiếm đơn giản, người dùng đã có thể truy cập đến vô vàn các khoá học, sách tham khảo, bài giảng online khác nhau. Tuy nhiên, với lượng thông tin ngày càng nhiều và đa dạng như hiện nay, không phải khoá học, tài liệu nào cũng phù hợp với mục đích và trình độ học của người dùng. Việc theo học một khoá học không phù hợp sẽ dẫn đến việc người học mất dần hứng thú, động lực học, gây ra tốn kém thời gian mà hiệu quả thu được là không cao.

Để giải quyết bài toán tìm kiếm và lựa chọn thông tin cần thiết phù hợp với nhu cầu người dùng, các hệ thống thông tin thường tích hợp một hệ lọc để đưa ra chỉ những thông tin mà người dùng có thể quan tâm. Hệ thống này được gọi là hệ thông tư vấn, hay hệ gợi ý (Recommender System). Hệ thống tư vấn dựa trên các thông tin thu thập được từ người dùng, phân tích xử lý và đối chiếu với cơ sở tri thức, từ đó đưa ra được những thông tin hữu dụng giúp cho người dùng đạt được mục đích của mình. Hiện nay trên thế giới đã có rất nhiều hệ thống tư vấn được tích hợp ứng dụng trong nhiều lĩnh vực khác nhau như thương mai điện tử, phim ảnh, âm nhạc, sách,... Tuy nhiên rất ít trong số đó dùng cho mục đích tư vấn tài liệu học, và hầu hết các hệ thống giáo dục trực tuyến không được tích hợp chức năng tư vấn.

Để giải quyết vấn đề trên, đồ án	 này đề xuất một hệ thống tư vấn tài liệu học tiếng Anh cho người dùng trên điện thoại di động giúp cho người dùng có thể lựa chọn được tài liệu học phù tiếng Anh phù hợp với bản thân mình. Hệ thống được xây dựng dựa vào việc thu thập thông tin về trình độ và nguyện vọng của người dùng, đồng thời dựa vào đánh giá của người dùng về những kết quả tư vấn. Qua đó xây dựng được hồ sơ người dùng và tư vấn ra những kết quả phù hợp với họ. 

\section{Định hướng giải quyết}

Hệ thống tư vấn sẽ được xây dựng theo mô hình client-server. Trong đó client sẽ là ứng dụng điện thoại di động được viết trên nền tảng Android đóng vai trò một giao diện tương tác, thu thập và xử lý thông tin người dùng và trả về kết quả tư vấn. Server lữu trữ trên nền tảng Firebase của Google sẽ đóng vai trò là hệ cơ sở tri thức, lưu giữ thông tin hồ sơ người dùng, tài liệu tiếng Anh và mạng lưới từ khoá (keyword network) sử dụng thực hiện tư vấn. 

Trên client, người dùng sẽ được yêu cầu cung cấp về thông tin trình độ cũng như mong muốn học của họ, cụ thể là : \\\\
<trình độ>
\begin{itemize}  
        \item Thời điểm bắt đầu học 
        \item Trình độ đọc hiểu
        \item Vốn từ vựng
        \item Vốn ngữ pháp 
    \end{itemize}
<nguyện vọng>  
\begin{itemize}  
        \item Mục đích học của người dùng
    \end{itemize}
    
Từ tập câu hỏi trên và xây dựng thành profile người dùng. Dựa trên profile này, hệ thống tiến hành context - match giữa hồ sơ người dùng và thuộc tính của từng tài liệu và chấm điểm độ phù hợp, sau đó hiện kết quả cho người dùng về khoá học, tài liệu tương ứng với trình độ và nguyện vọng của họ. Người dùng sau đó sẽ tiến hành đánh giá xem các kết quả tư vấn trả về có phù hợp với họ hay không. Hồ sơ người dùng sẽ có sự thay đổi dựa trên các đánh giá đó. Cứ tiếp tục như vậy, các kết quả tư vấn tiếp theo sẽ càng ngày chính xác đúng với nhu cầu cá nhân của người dùng hơn.

