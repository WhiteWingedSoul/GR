\def\baselinestretch{1}
\chapter{Kết luận và hướng phát triển} \label{KetLuan}

\section{Các kết quả đã đạt được}
Đồ án đã đạt được các kết quả sau đây:
\begin{itemize}
	\item Nắm được kiến thức về thuật toán Context-Matching và áp dụng được vào trong tư vấn tài liệu.
	\item Nắm được kiến thức về Kansei Engineering và áp dụng được vào trong cải thiện kết quả tư vấn sao cho phù hợp với từng cá nhân người dùng.
	\item Mô hình bài thi tương tác tuỳ biến cho ra kết quả đo trình độ người dùng với độ chính xác ngang bằng bài kiểm tra thông thường dù chỉ cần sử dụng số lượng câu hỏi ít hơn.
	\item Xây dựng được ứng dụng thử nghiệm cho ra kết quả tư vấn tài liệu tương đối chính xác.
	\item Ứng dụng xây dựng được là một công cụ hữu ích hỗ trợ cho việc học E-learning. Có tiềm năng phát triển thành module hỗ trợ việc giảng dạy cho các trung tâm/trường học dạy tiếng Anh qua mạng.
\end{itemize}

\section{Những hạn chế còn tồn đọng}
Tuy nhiên, đồ án không thể tránh khỏi các thiết sót còn cần giải quyết :
\begin{itemize}
	\item Logic thuật toán được cài đặt và xử lý trực tiếp trên thiết bị, dẫn đến hiệu năng chưa cao. Các bước xử lý tốn tương đối nhiều thời gian làm ứng dụng chạy chưa được mượt.
	\item Khối lượng tài liệu tư vấn còn ít, chưa dẫn đến các kết quả tư vấn của mỗi người dùng khác nhau vẫn còn tương đối giống nhau.
	\item Câu hỏi kiểm tra trình độ tiếng Anh chưa thực sự phản ánh được đúng trình độ người dùng do kiến thức hiểu biết về việc giảng dạy tiếng Anh còn hạn chế.
\end{itemize}

\section{Định hướng phát triển trong tương lai}
Đề tài "Xây dựng hệ thống tư vấn tài liệu học tiếng Anh E-Learning" trong khuôn khổ đồ án này chỉ dừng lại ở mức độ tìm hiểu, nghiên cứu và bước đầu xây dựng ứng dụng thử nghiệm thuật toán. Để có thể đưa vào triển khai trong môi trường thực tế, ứng dụng cần được tiếp tục hoàn thiện và phát triển. Sau đây là một số hướng đi đề xuất để phát triển ứng dụng trong tương lại:
\begin{itemize}
	\item Xây dựng hệ thống trên nền tảng Web, chuyển các tác vụ xử lý nặng lên backend của Server.
	\item Bổ sung các câu hỏi kiểm tra phù hợp hơn, thêm bài kiểm tra các trình độ Nói, Nghe Hiểu, Viết Văn. 
	\item Bố sung tài liệu tiếng Anh có chọn lọc, thêm các tài liệu dưới dạng khoá học online, bài giảng ..v..v...
	\item Bổ sung các chức năng quản trị hệ thống cho người quản trị như quản lý người dùng, thêm/sửa/xoá tài liệu, bài kiểm tra, log hành vi của người dùng ..v..v... 
\end{itemize}
