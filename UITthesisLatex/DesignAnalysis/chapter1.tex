% \pagebreak[4]
% \hspace*{1cm}
% \pagebreak[4]
% \hspace*{1cm}
% \pagebreak[4]

\chapter{Phân tích thiết kế xây dựng hệ thống} \label{design-analysis}

\ifpdf
    \graphicspath{{DesignAnalysis/Chapter1Figs/PNG/}{DesignAnalysis/Chapter1Figs/PDF/}{DesignAnalysis/Chapter1Figs/}}
\else
    \graphicspath{{DesignAnalysis/Chapter1Figs/EPS/}{DesignAnalysis/Chapter1Figs/}}
\fi

\section{Phân tích tổng quát hệ thống}

Các yêu cầu của hệ thống:
\begin{itemize}
	\item Cung cấp các phương pháp để xác định được trình độ, nguyện vọng và các yếu tố ưu tiên khác trong việc học tiếng Anh của người dùng thông qua giao diện đơn giản, dễ tương tác. Qua đó, đưa ra được các kết quả tư vấn là tài liệu, sách, video, bài giảng Tiếng Anh..v..v... tương ứng.
	\item Cung cấp giao diện quản lý cho quản trị viên dễ dàng thực hiện các thao tác quản lý thông tin người dùng, quản lý hệ cơ sở tri thức gồm tập câu hỏi kiểm tra và tài liệu tiếng Anh.
\end{itemize}
Qua việc khảo sát các hệ thống tư vấn đã và đang được triển khai, đồng thời để giải quyết các yêu cầu đặt ra ở trên, hệ thống đề xuất xây dựng trong đề tài này sẽ có cấu trúc gồm các thành phần sau:

\begin{itemize}
	\item \textbf{Module xác định trình độ:} có nhiệm vụ xác định trình độ của người sử dụng, thông qua việc thực hiện bài kiểm tra General English Test . Sử dụng kĩ thuật Computerized Adaptive Testing, các câu hỏi đưa ra cho người dùng sẽ được tuỳ biến sao cho độ khó phù hợp với năng lực của người dùng. Nhờ vậy, số lượng câu hỏi mà người dùng cần trả lời để xác định được trình độ của họ là ít hơn bài kiểm tra truyền thống, song vẫn cho ra kết quả chính xác như tương tự. Kết quả của bước này sẽ cho ra User level bao gồm trình độ đọc hiểu, vốn từ vựng và vốn ngữ pháp.
	\item \textbf{Module xác định nguyện vọng:} có nhiệm vụ nhận input về nguyện vọng từ người dùng, cụ thể là chủ đề mà người dùng muốn học. Có thể đưa ra các gợi ý cho người dùng về các chủ đề phổ biến. Kết quả bước này sẽ cho ra User preference là các chủ đề người dùng muốn học dưới dạng keyword.
	\item \textbf{Context-matching:} thực hiện nhận thông tin User level và User preference tổng hợp thành User profile. Sau đó sử dụng thuật toán Context-matching tiến hành matching với profile tài liệu và trả về những kết quả có độ khớp cao nhất.
	\item \textbf{Kansei:}
	\item \textbf{Hệ cơ sở tri thức:} sử dụng cơ sở dữ liệu online của Firebase làm cơ sở tri thức cho hệ thống. Nhiệm vụ của nó là trao đổi thông tin với client, cập nhập thông tin mới đảm bảo tính đồng bộ cho toàn hệ thống. \\Dữ liệu được lưu trữ bao gồm:
		\begin{itemize}
			\item Dữ liệu câu hỏi và đáp án General English Test
			\item Thông tin người dùng : id, loại người dùng, trình độ, nguyện vọng.
			\item Dữ liệu tài liệu học Tiếng Anh: tên, loại tài liệu, tác giả, miêu tả, nội dung ..v..v... và profile của tài liệu dưới dạng một tập keyword.
		\end{itemize}
\end{itemize}

Sau đây là mô hình kiến trúc của ứng dụng:

\begin{figure}[H]
  \begin{center}
    %\leavevmode
    \ifpdf
      \includegraphics[scale=0.5]{appflow}
    \else
      \includegraphics[scale=0.5]{appflow}
    \fi
    \caption{Mô hình kiến trúc ứng dụng}
    \label{Appflow}
  \end{center}
\end{figure}



\section{Xây dựng hồ sơ người dùng}
User profile thể hiện trình độ và nguyện vọng của người dùng. Những thông tin này sẽ được dùng trong quá trình matching với các tập dữ liệu bằng thuật toán Context-matching . \\
Một user profile sẽ gồm có các thuộc tính sau:
\begin{itemize}
\item Thời điểm bắt đầu học
\item Trình độ nghe hiểu
\item Trình độ đọc hiểu
\item Trình độ nói
\item Trình độ viết văn
\item Mục đích học của người dùng\\
\end{itemize}
Hệ thống sẽ đưa ra tập câu hỏi với các đáp án cho sẵn để xác định profile người dùng :
\begin{itemize}
\item Bạn học tiếng Anh được bao lâu rồi ?
\item Trình độ đọc hiểu của bạn thế nào ?
\item Bạn viết tiếng Anh tốt đến mức nào ?
\item Bạn giao tiếp bằng tiếng Anh có tốt không ?
\item Khả năng nghe tiếng Anh của bạn như thế nào ?
\item Bạn mong muốn được cải thiện những kĩ năng nào ?	
\end{itemize}

Câu hỏi 1-5 cho ta trình độ của người dùng, mỗi câu hỏi có 5 đáp án tương ứng với các mức: mới học, cơ bản, trung cấp, nâng cao, thành thạo. Dựa vào câu trả lời của người dùng, ta xác định được thuộc tính trong user profile về các kĩ năng đọc, nghe, viết, nói.

Thêm vào đó, để đánh giá chung năng lực của người dùng. Mỗi đáp án được gán với 1 mức điểm đánh giá năng lực tương ứng : a - 0, b - 0.3, c - 0.6, d - 0.85, e - 1. Gọi d là điểm đánh giá của từng câu hỏi, ta có d = điểm đánh giá đáp án x trọng số câu hỏi. Từ đây ta tính được điểm đánh giá trình độ của người dùng bằng tổng d1+d2+d3+d4+d5. Trình độ người dùng sẽ được phân loại vào các mức: mới học (điểm=0), sơ cấp (điểm<=0.3), trung cấp(điểm<=0.6), cao cấp(điểm<=0.85) và thành thạo(điểm<=1)

Vd. Anh A học tiếng Anh được trên 5 năm, anh A có thể đọc được tài liệu tiếng Anh chuyên ngành, viết được đoạn văn ngắn bằng tiếng Anh, nghe hiểu hội thoại tự nhiên trong cuộc sống và giao tiếp như người bản xứ. Điểm trình độ của anh A sẽ là : 0.8*0.1 + 1*0.225+0.6*0.225+0.85*0.225+1*0.225 = 0.85625. => Trình độ của anh A nằm ở mức độ thành thạo.

Câu hỏi cuối cho ta nguyện vọng học của người dùng. Nguyện vọng sẽ ảnh hưởng đến việc chọn cuốn sách dựa trên kĩ năng mà nó cung cấp.
\section{Xây dựng mạng từ khoá}


